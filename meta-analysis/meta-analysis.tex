\title{Meta-analysis Cheat Sheet}
\author{Joe Hilgard}
\documentclass{article}
\begin{document}
\maketitle

\begin{center}
Rules of meta-analysis:
\end{center}

\begin{enumerate}
\item All effect sizes must be relevant and competent tests of the hypothesis.
\item All effect sizes must involve a single degree of freedom in the numerator (e.g., a two-sample t-test or the interaction term from a $2\times2$ ANOVA).
	\begin{enumerate}
	\item If you have more than one degree of freedom in the numerator, make the appropriate contrast(s) that have one degree of freedom.
	\end{enumerate}
\item Each row is one effect size.
	\begin{enumerate}
	\item Multiple contrasts within one study have the same study name, same outcome name, different contrast name.
	\item Multiple outcomes within one study have the same study name, same contrast name, different outcome name.
	\end{enumerate}
\end{enumerate}

\section{Cohen's d}
Cohen's d is an effect size for the difference between two categorical groups in some continuous outcome.
When running a $2\times2$ ANOVA, Cohen's d is still applicable because the interaction term is essentially comparing two groups. A1 and B2 get +1 contrast weights and A2 and B1 get -1 contrast weights. You end up with the test $(A1 + B2) - (A2 + B1)$.

How to calculate Cohen's $d$.
\begin{equation}
d = \frac{\bar{X}_{1} - \bar{X}_{2}}{S_{within}}
\end{equation}
$\bar{X}_{1}$ is the mean of group 1. $\bar{X}_{2}$ is the mean of group 2. $S_{within}$ is the pooled standard deviation. You can get the pooled standard deviation function by loading the {\tt hilgard} library in R and using the {\tt pool.sd()} function.


Using {\tt escalc}
Load the {\tt metafor} library in R.
escalc(method = "SMD", m1i = 10.4, m2i = 8.1, sd1i = 2, sd2i = 1.6, n1i = 50, n2i = 45)
Note that "SMD" is case-sensitive. It will not work if you type in "smd".
In the case of a complex contrast or an interactionterm, you may need to pool some groups to reduce things to just two means, two SDs, and two sample sizes.
Take the weighted average of the cell means.
weighted.mean(x = c(11.3, 15.2), w = (25, 26)
Make the pooled sd of the cells.
pool.sd(sds = c(2.1, 1.8), ns = 25, 26)


\end{document}